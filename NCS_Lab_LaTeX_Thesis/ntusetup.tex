% !TeX root = ./main.tex

% --------------------------------------------------
% 資訊設定(Information Configs)
% --------------------------------------------------

\ntusetup{
  university*   = {National Taiwan University},
  university    = {國立臺灣大學},
  college       = {電機資訊學院},
  college*      = {College of Electrical Engineering and Computer Science},
  institute     = {電機工程學系},
  institute*    = {Department of Electrical Engineering},
  title         = {國立臺灣大學碩博士畢業論文模版},
  title*        = {National Taiwan University (NTU) \\ Thesis/Dissertation Template in \LaTeX},
  author        = {朱雁丞},
  author*       = {Yen-Cheng Chu},
  ID            = {R10921008},
  advisor       = {連豊力},
  advisor*      = {Feng-Li Lian},
  date          = {2022-08-14},         % 若註解掉,則預設為當天
  oral-date     = {2022-08-14},         % 若註解掉,則預設為當天
  DOI           = {10.5566/NTU2018XXXXX},
  keywords      = {LaTeX, 中文, 論文, 模板},
  keywords*     = {LaTeX, CJK, Thesis, Template},
}

% --------------------------------------------------
% 加載套件(Include Packages)
% --------------------------------------------------

%\usepackage[sort&compress]{natbib}      % 參考文獻, use biblatex instead
\usepackage{amsmath, amsthm, amssymb}   % 數學環境
% \usepackage{ulem, CJKulem}              % 下劃線、雙下劃線與波浪紋效果
\usepackage{CJKulem}                    % 下劃線、雙下劃線與波浪紋效果
\usepackage{booktabs}                   % 改善表格設置
\usepackage{multirow}                   % 合併儲存格
\usepackage{diagbox}                    % 插入表格反斜線
\usepackage{array}                      % 調整表格高度
\usepackage{longtable}                  % 支援跨頁長表格
\usepackage{paralist}                   % 列表環境


\usepackage{lipsum}                     % 英文亂字
\usepackage{zhlipsum}                   % 中文亂字

\usepackage[normalem]{ulem}             % to strike the words

% --------------------------------------------------
% 套件設定(Packages Settings)
% --------------------------------------------------

\newtheorem{assumption}{Assumption}
\newtheorem{theorem}{Theorem}
\newtheorem{lemma}{Lemma}
\newtheorem{remark}{Remark}
\newtheorem{proposition}{Proposition}
\newtheorem{definition}{Definition}
